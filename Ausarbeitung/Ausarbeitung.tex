% Diese Zeile bitte -nicht- aendern.
\documentclass[course=erap]{aspdoc}

%%%%%%%%%%%%%%%%%%%%%%%%%%%%%%%%%
%% done: Ersetzen Sie in den folgenden Zeilen die entsprechenden -Texte-
%% mit den richtigen Werten.
\newcommand{\theGroup}{108} % Beispiel: 42
\newcommand{\theNumber}{A329} % Beispiel: A123
\author{Cara Dickmann \and Thua Duc Nguyen \and Eslam Nasrallah}
\date{Wintersemester 2022/23} % Beispiel: Wintersemester 2019/20
%%%%%%%%%%%%%%%%%%%%%%%%%%%%%%%%%

% Diese Zeile bitte -nicht- aendern.
\title{Gruppe \theGroup{} -- Abgabe zu Aufgabe \theNumber}

\begin{document}
\maketitle

\section{Einleitung}


\section{Lösungsansatz}

\subsection{Beweis der Polynomdarstellung}
\begin{align*}
    \pi &= 2 + \sum_{n=1}^{\infty} {\frac{n!^2 \cdot 2^{n+1} }{(2n+1)!}} 
    = \sum_{n=0}^{\infty} {\frac{n!^2 \cdot 2^{n+1} }{(2n+1)!}} 
    = \sum_{n=0}^{\infty} {\frac{2}{1} \cdot \frac{n!^2 \cdot 2^n}{(2n+1)!}} \\
    &= \sum_{n=0}^{\infty} {\frac{2}{1} \cdot \frac{(2 \cdot 1^2) \cdots (2 n^2)}{1 \cdots (2n+1)}}
    = \sum_{n=0}^{\infty} {\frac{2}{1} \cdot \prod_{k=1}^{n} {\frac{2k^2}{2k \cdot (2k+1)}}} \\
    &= \sum_{n=0}^{\infty} {\frac{2}{1} \cdot \prod_{k=1}^{n} {\frac{k}{2k+1}}}
    = \sum_{n=0}^{\infty} {\frac{a(n)}{b(n)} \cdot \prod_{k=1}^{n} {\frac{p(k)}{q(k)}}} \\
\label{eq:first}
\end{align*}






% TODO: Je nach Aufgabenstellung einen der Begriffe wählen
\section{Korrektheit/Genauigkeit}


\section{Performanzanalyse}


\section{Zusammenfassung und Ausblick}

% TODO: Fuegen Sie Ihre Quellen der Datei Ausarbeitung.bib hinzu
% Referenzieren Sie diese dann mit \cite{}.
% Beispiel: CR2 ist ein Register der x86-Architektur~\cite{intel2017man}.
\bibliographystyle{plain}
\bibliography{Ausarbeitung}{}

\end{document}
